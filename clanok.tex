% Metódy inžinierskej práce

\documentclass[10pt,twoside,,a4paper]{article}

\usepackage[slovak]{babel}
%\usepackage[T1]{fontenc}
\usepackage[IL2]{fontenc} % lepšia sadzba písmena Ľ než v T1
\usepackage[utf8]{inputenc}
\usepackage{graphicx}
\usepackage{url} % príkaz \url na formátovanie URL
\usepackage{hyperref} % odkazy v texte budú aktívne (pri niektorých triedach dokumentov spôsobuje posun textu)

\usepackage{cite}
%\usepackage{times}

\nocite{*}

\pagestyle{headings}

\title{Použitie odporúčacích algoritmov v zdravotníctve založených na predošlých zdravotných dátach pacienta a iných lokálnych faktorov za účelom individualizácie zdravotnej starostlivosti\thanks{Semestrálny projekt v predmete Metódy inžinierskej práce, ak. rok 2024/25, vedenie: Ivan Kapustík}} % meno a priezvisko vyučujúceho na cvičeniach

\author{Lukáš Náhlik\\[2pt]
	{\small Slovenská technická univerzita v Bratislave}\\
	{\small Fakulta informatiky a informačných technológií}\\
	{\small \texttt{xnahlik@stuba.sk}}
	}

\date{\small \today}



\begin{document}

\maketitle

\begin{abstract}
Použitie odporúčacích algoritmov v zdravotníctve predstavuje významný krok smerom k personalizovanej zdravotnej starostlivosti, ktorý môže zásadne ovplyvniť spôsob, akým sú diagnostikované a liečené ochorenia. Tento článok sa zameriava na implementáciu algoritmov, ktoré využívajú predošlé zdravotné údaje pacienta, ako sú anamnéza, genetické faktory, záznamy o predchádzajúcich liečbach, a iné demografické údaje, v kombinácii s lokálnymi faktormi, ako je geografická dostupnosť zdravotných služieb, sociálno-ekonomické podmienky, a dokonca aj klimatické vplyvy. Cieľom týchto algoritmov je zlepšiť rozhodovanie v procese diagnostiky, liečby a preventívnej starostlivosti tým, že poskytujú lekárom aj pacientom odporúčania šité na mieru individuálnym potrebám.

V článku sa podrobne venujeme analýze, ako môžu tieto algoritmy prispieť k efektívnejšej a presnejšej diagnostike, rýchlejšiemu výberu optimálnej liečby a zlepšeniu preventívnych opatrení. Ďalej skúmame, ako môžu pomôcť riešiť nerovnosti v prístupe k zdravotnej starostlivosti, pričom berieme do úvahy lokálne špecifiká a individuálne potreby pacientov v rôznych regiónoch. Súčasne diskutujeme o výzvach spojených s použitím odporúčacích systémov v zdravotníctve, vrátane otázok ochrany súkromia, bezpečnosti citlivých zdravotných dát, transparentnosti algoritmov a rizík spojených s ich nadmerným používaním.

Záverečná časť článku sa venuje perspektívam využitia odporúčacích algoritmov v klinickej praxi a ich potenciálu zlepšiť kvalitu a efektivitu zdravotnej starostlivosti na globálnej úrovni. 
\end{abstract}
\newpage



\section{Úvod}
\subsection{Kontext a význam individualizácie zdravotnej starostlivosti}
zdravotna starostlivost
\cite{RSH}
\subsection{Prehľad technológií používaných v zdravotníctve, vrátane odporúčacích algoritmov}
\subsection{Cieľ článku a hlavné oblasti skúmania}
\section{Odporúčacie algoritmy v zdravotníctve}
\subsection{Definícia a typy odporúčacích systémov}
\subsection{Spôsoby využitia v zdravotníckom kontexte (diagnostika, liečba, prevencia)}
\subsection{Príklady úspešných implementácií v praxi}
\section{Zdroje dát pre odporúčacie algoritmy}
\subsection{Predošlé zdravotné dáta pacienta (anamnéza, genetika, liečby)}
\subsection{Lokálne faktory (geografická dostupnosť, socio-ekonomické podmienky, prostredie)}
\subsection{Integrácia rôznych zdrojov dát pre optimalizáciu algoritmov}
\section{Individualizácia zdravotnej starostlivosti pomocou odporúčacích algoritmov}
\subsection{Ako algoritmy personalizujú zdravotnú starostlivosť}
\subsection{Vplyv na diagnostiku a liečbu (personalizované liečebné plány)}
\subsection{Úloha algoritmov v preventívnej medicíne}
\section{Výzvy a riziká pri použití odporúčacích algoritmov}
\subsection{Ochrana súkromia a bezpečnosť zdravotných dát}
\subsection{Etické otázky a transparentnosť algoritmov}
\subsection{Možné chyby a riziká nadmerného spoliehania sa na algoritmy}
\section{Potenciál odporúčacích algoritmov v budúcnosti zdravotnej starostlivosti}
\subsection{Možnosti ďalšieho vývoja technológie}
\subsection{Potenciálne vplyvy na efektivitu a kvalitu zdravotnej starostlivosti}
\subsection{Perspektívy globálnej aplikácie a redukcia nerovností v zdravotníctve}
\section{Záver}
\subsection{Zhrnutie hlavných bodov}
\subsection{Budúce smery výskumu a vývoja v oblasti odporúčacích systémov v zdravotníctve}
\subsection{Vplyv na prax a pacientov}
\section{Referencie}
\subsection{Zoznam použitých vedeckých a technických zdrojov, ktoré podporujú tvrdenia v článku}










% týmto sa generuje zoznam literatúry z obsahu súboru literatura.bib podľa toho, na čo sa v článku odkazujete
\bibliography{literatura}
\bibliographystyle{alpha} % prípadne alpha, abbrv alebo hociktorý iný
\end{document}
